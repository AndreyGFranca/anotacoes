\documentclass[10pt,a4paper]{article}


\usepackage[utf8]{inputenc}
\usepackage{amsmath}
\usepackage{amsfonts}
\usepackage{amssymb}
\usepackage{mdframed}
\usepackage{amsthm}
\usepackage{setspace}
\usepackage[lastexercise]{exercise}

\author{Andrey França}

%theorems
\theoremstyle{definition}
\newtheorem*{eg}{Exemplo}
\newtheorem*{ex}{Exercício}
\newtheorem*{sol}{Solução}
\newtheorem*{df}{Definição}
\newtheorem*{obs}{Observação}
\newtheorem*{imp}{Implementação}

%commands
%\renewcommand{\baselinestretch}{1.3} 


\title{Anotações Calculo}

\begin{document}
\maketitle
\tableofcontents
\newpage


\section{Funções}
Em geral, consideramos as funções para as quais D e E são conjuntos de números reais. O conjunto D é chamado domínio da função. O número f (x) é o valor de f em x e é lido “ f de x”. A imagem de f é o conjunto de todos os valores possíveis de f (x) obtidos quando x varia por todo o \textbf{domínio}. O símbolo que representa um número arbitrário no domínio de uma função f é denominado uma \textbf{variável independente}. Um símbolo que representa um número na \textbf{imagem} de f é denominado uma \textbf{variável dependente}.
\begin{df}
	Uma \textbf{função} f é uma lei que associa, a cada elemento x em um conjunto D, exatamente um elemento, chamado f(x), em um conjunto E.
\end{df}
\subsection*{Representando uma função}
\begin{multicols}{2}
\begin{tikzpicture}[xscale=0.75,yscale=0.75]
	\begin{axis}[%
		samples = 50,
		axis x line = center,
		axis y line = center,
		xlabel = {$x$},
		ylabel = {$y$},
		ticks=none,
	]
	\addplot[black] {x^2}  node[pos=.70,right] {$y=x^{2}$};
	\end{axis}
\end{tikzpicture}

\columnbreak
É possível representar uma função de quatro maneiras:
\begin{itemize}
	\item[•]verbalmente
	\item[•]numericamente
	\item[•]visualmente
	\item[•]algebricamente
\end{itemize}
\end{multicols}


\begin{df}
	Uma função $f$ é chamada crescente em um intervalo \textit{I} se
	
	\begin{center}
		$f(x_{1} < f(x_{2}))$ quando $x_{1} < x_{2}$ em \textit{I}.
	\end{center}

	É denominada decrescente em I se

	\begin{center}
		$f(x_{1} > f(x_{2}))$ quando $x_{1} > x_{2}$ em \textit{I}.
	\end{center}
\end{df}

\section{Limites}


\end{document}