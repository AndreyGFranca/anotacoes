\documentclass[10pt,a4paper]{article}

\usepackage[utf8]{inputenc}
\usepackage{amsmath}
\usepackage{amsfonts}
\usepackage{amssymb}
\usepackage{mdframed}
\usepackage{amsthm}
\usepackage{setspace}
\usepackage[lastexercise]{exercise}

\author{Andrey França}

%theorems
\theoremstyle{definition}
\newtheorem*{eg}{Exemplo}
\newtheorem*{ex}{Exercício}
\newtheorem*{sol}{Solução}
\newtheorem*{df}{Definição}
\newtheorem*{obs}{Observação}
\newtheorem*{imp}{Implementação}

\newcounter{x}\setcounter{x}{1}
\newtheorem{inneraxiom}{Axiom}
\newenvironment{axiom}[1]
  {\renewcommand\theinneraxiom{\arabic{x} (#1)}\inneraxiom\stepcounter{x}}
  {\endinneraxiom}

%commands
%\renewcommand{\baselinestretch}{1.3} 


\title{Anotações Teoria da Probilidade}


\begin{document}
\maketitle
%\small
\tableofcontents
\newpage

\section{Espaço Amostral}
\begin{mdframed}[linewidth=0.6pt]
\textbf{Remark}\\
	Chamamos de espaço amostral, e indicamos por $\Omega$, um conjunto formado por todos os
	resultados possíveis de um experiemento aleatório.
\end{mdframed}

\begin{eg}
	Um experimento jogar um \textit{dado} tem seis resultados possíveis: 1, 2, 3, 4, 5, 6. Logo, o espaço amostral é $\Omega$ = $\lbrace1, 2, 3, 4, 5, 6\rbrace$.
\end{eg}
\begin{eg} 
	Lançar uma moeda duas vezes e observar a sequência de caras e coroas.
	
	$\Omega$ = $\lbrace (K,K), (K,C), (C,K), (C,C)\rbrace$.
\end{eg}

\subsection{Exercícios}
Dar o Espaço Amostral para cada exercício abaixo:\\
\begin{eg}
	Uma letra é escolhida entre as letras da palavra PROBABILIDADE.
	
	\textit{Sol.} O espaço amostral é qualquer uma das letras da palavra. Portanto 
	
	$\Omega = \lbrace P, R, O, B, I, L, D, A, E \rbrace$
\end{eg}

\begin{eg}
	Três pessoas A, B, C são colocadas numa fila e observa-se a disposição das mesmas.

	\textit{Sol. $\Omega = \lbrace (CAB), (BCA), (ACB), (CBA), (BAC) \rbrace$}
\end{eg}

\section{Eventos}
Consideremos um experimento aleatório, cujo espaço amostral é $\Omega$. Chamaremos de \textit{evento} todo subconjunto de $\Omega$. Em geral, indicamos um evento por uma letra maíuscula do alfabeto: A, B, C, D, ... X, Y, Z.

\begin{eg} 
	Um dado é lançado, e observa-se o número da face de cima.
	$\Omega = \lbrace 1, 2, 3, 4, 5, 6 \rbrace$\\

	Eis alguns eventos\\
	A: Ocorrência de número ímpar. A = $\lbrace1, 3, 5 \rbrace$\\
	B: Ocorrência de número primo. B = $\lbrace2, 3, 5$\\
	C: Ocorrência de número menor que 4. C = $\lbrace1, 2, 3 \rbrace$\\
\end{eg}
\begin{eg}
	Uma moeda é lançada 3 vezes, e observa-se a sequencia de caras e coroas.\\
	A: ocorrência de cara no primeiro lançamento\\
	A = $\lbrace(K,K,K), (K,C,K), (K,K,C), (K,C,K) \rbrace$
\end{eg}

\begin{mdframed}[linewidth=0.6pt]
	\textbf{Remark}\\
	Podemos, e devemos, realizar todas as operações de teoria dos conjuntos nos conjuntos formados 
	pelos eventos. Isto é, união, intersecção, ou seja, existem eventos que acontecem se um 
	\textsc{ou} outros ocorrerem, e se um \textsc{e} outro ocorrerem.
\end{mdframed}

\subsection{Exercícios}
\paragraph{E1 - }Uma moeda e um dado são lançados. Seja\\
$\Omega = \lbrace(K,1), (K,2),(K,3),(K,4),(K,5),(K,6)$\\
$    (C,1),(C,2),(C,3),(C,4),(C,5),(C,6), \rbrace$\\
Descreva os eventos:\\
a)A: ocorre cara,\\
b)B: ocorre número par,\\
\textit{Sol. TRIVIAL}

\section{Variável Aleatória}

\begin{df}
	Consideremos um experimento e $ \Omega $ o espaço amostral associado a esse experimento. \emph{ Uma função X, que associa a cada elemento $ \omega \in \Omega $ um número real, $ X(\omega) $,} é denominada variável aleatória (v.a.). Ou seja, variável aleatória é um característico numérico do resultado de um experimento.
\end{df}

\begin{eg}
	Considere três lançamentos independentes de uma moeda equilibrada. Seja C cara e K coroa. O espaço amostral deste experimento é S={(C,C,C); (C,C,K); (C,K,C); (K,C,C); (C,K,K); (K,C,K); (K,K,C); (K,K,K)}. Podemos definir a variável aleatória X: "número de caras obtidas nos três lançamentos". Por exemplo, temos que X((C,C,C)) = 3 e X((K,C,C))=2.
\end{eg}

\subsection{Função de Distribuição Cumulativa}

\begin{df}
A função de distribuição acumulada de uma variável aleatória X é uma função que a cada número real x associa o valor 
\[F(x)=\mathbb{P}\left(X\leq x\right), \quad x \in \Bbb{R}.\] 	

A notação $ \{X \leq x\} $ é usada para designar o conjunto $\lbrace \omega \in \Omega :  X(\omega) \leq x\} $, isto é, denota a imagem inversa do intervalo $ (-\infty,x] $ pela variável aleatória X. Com isso, podemos observar que a função de distribuição acumulada $ F $ tem como domínio os números reais $ (\Bbb{R}) $ e imagem o intervalo $ [0,1] $.

O conhecimento da função de distribuição acumulada é suficiente para entendermos o comportamento de uma variável aleatória. Mesmo que a variável assuma valores apenas num subconjunto dos reais, a função de distribuição é definida em toda a reta. Ela é chamada de função de distribuição acumulada, pois acumula as probabilidades dos valores inferiores ou iguais a x.
\end{df}



\section{Modelos Probabilísticos Discretos}
\subsection{Ensaios de Bernoulli}

Cosidere um experimento que consiste em uma sequência de ensaios ou tentaticas independentes, isto é, ensaios nos quais o resultado de uma ensaio não depende dos ensaios anteriores e nem dos ensaios posteriores. Em cada ensaio, podem ocorrer apenas dois resultados. Um deles que chamamos de sucesso(S) e outros que chamamos de fracasso (F). A probabilidade de ocorrer Sucesso é sempre p, e a probabilidade de ocorrer fracasso é sempre q = 1 - p.


Para um experimento que consiste na realização de $ n $ ensaios independentes de Bernoulli, o espaço amostral pode ser considerado como o conjunto de n-uplas, em que cada posição há um sucesso (S) ou uma falha (F).
\subsubsection{Exemplos de ensaios de Bernoulli}


1) Uma moeda é lançada 5 vezes. Cada lançamento é um ensaio, onde dois resultados podem ocorrer: cara ou coroa. Chamamos o Sucesso de cara, e o fracasso de coroa. Em cada ensaio p = $\frac{1}{2}$ e Fracasso q = $\frac{1}{2}$


Neste exemplo sejam os eventos:\\
$A_{1}$: ocorre cara no 1 lançamento, P($A_{1}$) = $\frac{1}{2}$\\
$A_{2}$: ocorre cara no 2 lançamento, P($A_{2}$) = $\frac{1}{2}$.\\
\vdots\\
$A_{5}$: ocorre cara no 5 lançamento, P($A_{5}$) = $\frac{1}{2}$.\\


Então o evento $A_{1} \cap A_{2} \cap \cdots \cap A_{5}$ corresponde ao evento sair cara nos 5 lançamentos, que é $\lbrace(K, K, K, K, K) \rbrace$

Como os 5 eventos são independentes, \\
\[P(A_{1} \cap A_{2} \cap A_{3} \cap A_{4} \cap A_{5}) = \frac{1}{2} \cdot \frac{1}{2}\cdot \frac{1}{2} \cdot \frac{1}{2} \cdot \frac{1}{2} = \frac{1}{32}\] 

Vamos supor agora o evento de sair exatamente 1 cara, isto é
\[\lbrace(K, C, C, C, C), (C, K, C, C, C), (C, C, K, C, C), (C, C, C, K, C), (C, C, C, C, K) \rbrace\]
Portanto a probabilidade é:
\[\frac{1}{32} + \frac{1}{32} + \frac{1}{32} + \frac{1}{32} + \frac{1}{32} = \frac{5}{32}\]
Como podemos facilmente perceber isto é a permutação de 5 elementos com 4 repetidos.
\begin{mdframed}[linewidth=0.6pt]
	\textbf{Remark}\\
	Dizemos que um evento A independe de B se:
	\begin{center}
		P(A$|$B) = P(A)
	\end{center}
	isto é:
	\[P(A|B) = \frac{P(A \cap B)}{P(B)} = \frac{P(B) P(A|B)}{P(A)} = \frac{P(B)P(A)}{P(A)} = 
	P(B)\]
\end{mdframed}

Se quisermos agora calcular a probabilidade de sair exatamente duas caras, teremos de calcular o números de quintuplas ordenadas, onde existem duas caras (K) e três coroas:
\[P^{2,3}_{5} = \frac{5!}{2!3!} = 10\]

\textit{Obs:} Aqui calculamos exatamente 5 ensaios de Bernoulli.

\subsection{Distribuição Binomial}


O que se conhece por Distribuição Binomial é justamente a generalização dos ensaios de Bernoulli. Vamos considerar uma sequencia de \textit{n} ensaios de Bernoulli. Seja \textit{p} a probabilidade de sucesso e \textit{q} a probabilidade de fracasso.

 Queremos calcular a \emph{probabilidade de $P_{k}$, da ocorrência de exatamente K sucessos, nos n ensaios.}

O evento \textit{Ocorrem K sucessos nos n ensaios} é formado por todas as enuplas ordenadas onde existem K sucessos(S) e n - K fracassos(F). O número de enuplas ordenadas nesta condição é:
\[
	P^{K, n-K}_{n} = \frac{n!}{K!(n-K)!} = \binom{n}{K}
\]

A probabilidade de cada enupla ordenada de K sucessos (S) e (n-K) Fracassos (F) é dada por:
\[
	\underbrace{p \cdot p \cdots p} \cdot \underbrace{q \cdot q \cdots q} = p^{K}\cdot q^{n-K}
\]

Logo, se cada enupla ordenada com exatamente K sucessos tem probabilidade de $p^{K} \cdot q^{n-K}$ e existem $\binom{n}{K}$ enuplas desse tipo, a probabilidade $P_{K}$ de exatamente K sucessos nos n ensaios será:
\[
	P_{K} = \binom{n}{K} \cdot p^{K} \cdot q^{n-K}
\]

\begin{eg}
	Um urna tem 4 bolas vermelhas (V) e 6 brancas (B). Uma bola é extraída, observa sua cor e reposta na urna. O experiemento é repetido 5 vezes. Qual a probabilidade de observarmos exatamente 3 vezes bola vermelha?


Em cada ensaio, consideremos como Sucesso o resultado "bola vermelha", e Fracasso o resultado "bola branca". Então:
	\[
		p = \frac{4}{10} = \frac{2}{5}, q = \frac{6}{10} = \frac{3}{5}, n = 5.	
	\]
	Estamos interessados na probabilidade $P_{3}$. Temos:
	
	\[
		P_{3} = \binom{5}{3}(\frac{2}{5})^{3} (\frac{2}{5})^{2} = \frac{5!}{3!2!} \cdot \frac{8}{125} \cdot \frac{9}{25} = \frac{725}{3125} \Rightarrow
		P_{3} = 0,2304
	\]
\end{eg}

\begin{eg}
	Numa cidade, 10\% das pessoas possuem carro de marca A. Se 30 pessoas são selecionadas ao acaso, com reposição, qual a probabilidade de exatamente 5 pessoas possuírem carro da marca A ?
	
	Em cada escolha de uma pessoa, consideremos os resultados:\\
	Sucesso: a pessoa tem carro da marca A.\\
	Fracasso: a pessoa não tem carro da marca A\\
	Então p = 0,1,    q = 0,9,    n = 30\\
	Estamos interessados em $P_{5}$. Temos:
	\[
		P_{5} = \binom{30}{5}(0,1)^{5} \cdot (0,9)^{25} \simeq 0,102	
	\]
\end{eg}

\subsubsection{Exercicios}
\begin{ex}
	Uma moeda é lançada 6 vezes. Qual a probabilidade de observarmos exatamente duas caras?
\end{ex}
\begin{sol}
	Vamos considerar que sair cara é Sucesso e sair coroa é Fracasso. Então temos que \textit{p} = 2, \textit{q} = 4 e \textit{n} = 6, portanto:
	\[
		\begin{aligned}
		P_{k} =& \binom{n}{K} \cdot p^{K} \cdot q^{n-K} =		\\
		& \binom{6}{2} \cdot (\frac{1}{2})^{6} \cdot (\frac{1}{2})^{4} =\\
		& \frac{6!}{2!4!} \cdot 0,15 \cdot 256=\\
		& 15 \cdot 0,15 \cdot 0,065=\\
		& \underline{0,14}
		\end{aligned}
	\]
\end{sol}

\subsection{Distrubuição de Poisson}
Em muitas situações nos deparamos com a situação em que o número de ensaios $ n $ é grande ($ n\rightarrow \infty $) e $ p $ é pequeno ($ p\rightarrow 0 $), no cálculo da função binomial, o que nos leva a algumas dificuldades, pois, como podemos analisar, para $ n $ muito grande e $ p $ pequeno, fica relativamente difícil calcularmos a probabilidade de $ k $ sucessos a partir do modelo binomial, isto é, utilizando a função de probabilidade

\[p(k)=\mathbb{P}(X=k)=\left(\begin{array}{c}n\\k\end{array}\right)p^k(1-p)^{n-k}.\] 	

Observamos que podemos reescrever a expressão acima da seguinte forma

\[\mathbb{P}(X=k)=\frac{n!}{k!(n-k)!}p^k\frac{n^k}{n^k}\left(1-\frac{np}{n}\right)^{n-k}=\frac{n!}{k!(n-k)!}\frac{(np)^k}{n^k}\left(1-\frac{np}{n}\right)^{n-k}\] 	

e, tomando $ \lambda = np $, segue que

\[\mathbb{P}(X = k)=\frac{n(n-1)\cdots(n-k+1)}{k!}\frac{\lambda^k}{n^k}\left(1-\frac{\lambda}{n}\right)^{n-k}=\frac{\lambda^k}{k!}1\left(1-\frac{1}{n}\right)\dots\left(1-\frac{k-1}{n}\right)\left(1-\frac{\lambda}{n}\right)^{n-k}.\] 	

Se tomarmos o limite quando $ n\rightarrow \infty $, obtemos que

\[\lim_{n\rightarrow \infty}1\left(1-\frac{1}{n}\right)\dots\left(1-\frac{k-1}{n}\right)=1\] 	

e

\[\lim_{n\rightarrow \infty}\left(1-\frac{\lambda}{n}\right)^{n-k}=\lim_{n\rightarrow \infty}\left(1-\frac{\lambda}{n}\right)^{n}=e^{-\lambda}\] 	

para $ k = 0,1,\ldots $ e $ e\approx 2,718 $.

Assim temos que 

\[\lim_{n\rightarrow \infty}\mathbb{P}(X=k)=\frac{e^{-\lambda}\lambda^k}{k!}.\] 	

Tal expressão é devida a Poisson e é muito utilizada para calcular probabilidades de ocorrências de defeitos "raros" em sistemas e componentes.

\begin{df}
	Uma variável aleatória discreta $ X $ segue a distribuição de Poisson com parâmetro $ \lambda $, $ \lambda \ \textgreater 0 $, se sua função de probabilidade for dada por

	\[\mathbb{P}(X=k)=\frac{e^{-\lambda}\lambda^k}{k!}.\] 	

	Utilizamos a notação $ X \sim \ \text{Poisson}(\lambda) $ ou $ X\sim \ \text{Po}(\lambda) $. O parâmetro $ \lambda $ indica a taxa de ocorrência por unidade medida.
\end{df}

\begin{mdframed}[linewidth=0.6pt]
\textbf{Remark}\\
	Uma área de oportunidadeé uma unidade contínua ou um intervalo de tempo, volume ou uma área tal que nela possa acontecermais de uma ocorrência de um evento. Exemplos:
	\begin{itemize}
		\item Defeitos na pintura de uma geladeira nova
		\item Número de falhas na rede em um determinado dia
		\item Número de pulgas no pêlo de um cachorro\\
	\end{itemize}
	A distribuição de Poisson é aplicada quando:
	\begin{itemize}  
		\item Você estiver interessado em contar o número de vezes em que um 
evento específico ocorre em uma determinada área de oportunidades. A área de oportunidades é definida pelo tempo, extensão, área de superfície e assim sucessivamente.
		\item A probabilidade de que um evento específico ocorra em uma 
determinada área de oportunidades é a mesma para todas as áreas 
de oportunidades.
		\item O número de eventos que ocorrem em uma determinada área de oportunidades é independente do número de eventos que ocorrem em qualquer outra área de oportunidades.
		\item A probabilidade de que dois ou mais eventos venham a ocorrer em uma determinada área de oportunidades se arpoxima de zero à medida que a área de oportunidades se torna menor.
	\end{itemize}

\end{mdframed}

\begin{eg}
	Suponha que, em média, 5 carros entrem em um estacionamento por minuto. Qual é a probabilidade de que em um dado minuto, 7 carros entrem?
\end{eg}
\begin{sol}
	Então, X = 7 e $\lambda$ = 5
	
	\[
		P(7) = \frac{e^{-\lambda}\lambda^{X}}{X!}	= \frac{e^{-7}\lambda^{5}}{7!} = 0,104
	\]	
	Portanto, há uma probabilidade de 10,4\% de que 7 carros entrem no estacionamento em um dado minuto.
\begin{flushright}
$\blacksquare$
\end{flushright}
\end{sol}


\subsubsection{Exercicios}
\begin{ex}
	Sabe-se que o número de acidentes de trabalho, por mês, em uma unidade de produção segue uma distribuição de Poisson, com uma média aritmética de 2,5  acidentes de trabalho por mês.
	\begin{itemize}
		\item[a)] Qual é a probabilidade de que em um determinado mês nenhum acidente de trabalho venha a ocorrer?
		\item[b)] De que pelo menos um acidente de trabalho venha a ocorrer?
	\end{itemize}
\end{ex}

\begin{sol}
	a) Com $\lambda$ = 2,5
	\[
		P(X = 0) = \frac{e^{-2,5}(2,5)^{0}}{0!} = \frac{1}{(2,71828)^{2,5}(1)} = 0,0821
	\]
	A probabildade de que em um determinado mês nenhum acidente de trabalho ocorra é 0,0821,  ou 8,21\%.
	
	
	b) 
	\[
		P(C \geq 1) = 1 - P(X = 0) = 1 - 0,0821 = 0,9179
	\]
	A probabilidade de que em um determinado mês haverá pelo menos um acidente de trabalho é  0,9179, ou 91,79\%.
\end{sol}

\begin{ex}
	Um departamento de polícia recebe em média 5 solicitações por hora.  Qual  a  probabilidade  de  receber  2 solicitações numa hora selecionada aleatoriamente? 
\end{ex}
\begin{sol}
	X = número designado de sucessos = 2 
	
	$\lambda$ = o número médio de sucessos num intervalo específico (uma hora) = 5 
	Então temos que:
	\[P(2) = \frac{e^{-\lambda}\lambda^{X}}{X!}	= \frac{e^{-5}5^{2}}{2!} = 0,084\]
\end{sol}

\begin{ex}
	A experiência passada indica que um número médio de 6 clientes por hora param para colocar gasolina numa bomba
	\begin{itemize}
		\item[a)] Qual é a probabilidade de 3 clientes pararem qualquer hora? 
		\item[b)] Qual é a probabilidade de 3 clientes pararem qualquer hora? 
	\end{itemize}
\end{ex}
\begin{sol}
	\begin{itemize}
		\item[a)]X = número designado de sucessos = 3 
		
		$\lambda$ = o número médio de sucessos num intervalo específico (uma hora) = 6 
		\[P(2) = \frac{e^{-\lambda}\lambda^{X}}{X!}	= \frac{e^{-6}6^{3}}{3!} = 0,089\]
		\item[b)]P(X $\leq$ 3) = P(0) + P(1) + P(2) + P(3) 
		
		Assim, P(X $\leq$ 3) = 0,00248 + 0,01488 + 0,04464 + 0,08928 = 0,15128 
	\end{itemize}
\end{sol}

\begin{ex}
	Suponhamos que em uma indústria farmacêutica 0,001$\%$ de um determinado medicamento sai da linha de produção somente com o excipiente, ou seja, sem nenhum princípio ativo. Qual a probabilidade de que em uma amostra de 4 mil medicamentos mais de 2 deles esteja somente com o excipiente.
\end{ex}
\begin{sol}
	Vamos calcular esta probabilidade usando a aproximação de poisson, pois além de ser uma aproximação muito boa neste caso é bem mais fácil de ser calculada.

Para usarmos a distribuição de Poisson primeiramente devemos encontrar o $ \lambda $, o qual é dado por $ \lambda=n\times p=4000\times 0,00001=0,04 $. Assim

\[\mathbb{P}(X\geq 2)=1-\mathbb{P}(X=0)-\mathbb{P}(X=1)=1-\displaystyle \frac{0,04^0e^{-0,04}}{0!}-\frac{0,04^1 e^{-0,04}}{1!}\approx 1-0,999221\approx 0,00078.\] 	

Assim a probabilidade de que existam mais de 2 medicamentos com apenas o excipiente é de 0,078%.
\end{sol}


\subsection{Distrubuição Geométrica}
\begin{quote}
	Geométrica (conta o número ensaios para se obter um sucesso.)
\end{quote}
\begin{df}
	Seja $ X $ a variável aleatória que fornece o número de falhas até o primeiro sucesso. A variável $ X $ tem distribuição Geométrica com parâmetro $ p $, $ 0 \ \textless \ p \ \textless \ 1 $, se sua função de probabilidade é dada por


\[\mathbb{P}\left(X=j\right)=(1-p)^jp, \quad j=0,1,\ldots\] 	

 

Usaremos a notação $ X \sim \ \text{Geo}(p) $.

O evento $ [X=j] $ ocorre se, e somente se, ocorrem somente falhas nos $ j $ primeiros ensaios e sucesso no $ (j+1) $-ésimo ensaio.
\end{df}

\begin{eg}
	Considere o experimento em que uma moeda viciada é lançada sucessivas vezes, até que ocorra a primeira cara. Seja $ X $ a variável aleatória que conta o número de coroas obtidos no experimento (ou seja, a quantidade de lançamentos anteriores a obtenção da primeira cara). Sabendo que a probabilidade de cara é de $ 0,4 $, qual é a probabilidade de $ \mathbb{P}(2\leq X\textless 4) $ e a probabilidade de $ \mathbb{P}(X\textgreater 1| X\leq 2) $.
\end{eg}

\begin{sol}
Observamos que

\[\mathbb{P}(2\leq X\textless 4)=\mathbb{P}(X=2)+\mathbb{P}(X=3)=0,6^2\cdot 0,4+0,6^3\cdot 0,4=0,2304.\] 	

Vamos calcular agora $ \mathbb{P}(X\textgreater 1| X\leq 2) $.

\[\mathbb{P}(X\textgreater 1| X\leq 2)=\frac{\mathbb{P}(X\textgreater 1\cap X\leq 2)}{\mathbb{P}(X\leq2)}=\frac{\mathbb{P}(X=2)}{\mathbb{P}(X=0)+\mathbb{P}(X=1)+\mathbb{P}(X=2)}=\frac{0,144}{0,784}=0,18367.\]
\end{sol}

\begin{eg}
	Um dado honesto é lançado sucessivas vezes até que apareça pela primeira vez a face 1. Seja $ X $ a variável aleatória que conta o número de ensaios até que corra o primeiro $ 1 $. Qual a probabilidade de obtermos $ 1 $ no terceiro lançamento.
\end{eg}
\begin{sol}
Como o dado é honesto, a probabilidade de, em um lançamento, obtermos qualquer face é igual a 1/6. Neste caso, a probabilidade de se obter a face $ 1 $ (sucesso) é $ 1/6 $ e a probabilidade de se obter qualquer outra face (fracasso) é $ 5/6 $. Podemos definir a variável aleatória


\[Y = \left\{\begin{array}{l}1, \ \text{se obtemos 1 no lançamento do dado}\\0, \ \text{caso contrário}\end{array}\right.\] 	

 

Neste caso, $ Y\sim \ \text{Bernoulli}(1/6) $ e, se definirmos $ X $ como sendo a variável que representa o número de lançamentos até a obtenção do primeiro sucesso (aparecimento da face 1), temos que $ X \sim \ \text{Geo}(1/6) $. Portanto, se estamos interessados no cálculo da probabilidade de obter $ 1 $ no terceiro lançamento, precisamos calcular $ \mathbb{P}(X = 3) $, ou seja,


\[\mathbb{P}(X=3)=(1-p)^2p=\left(1-\frac{1}{6}\right)^2\frac{1}{6}=\left(\frac{5}{6}\right)^2\frac{1}{6}=\frac{5^2}{6^3}\approx 0,09645.\]
\end{sol}
\begin{mdframed}[linewidth=0.6pt]
\textbf{Remark}

	A probabilidade de \textit{n} tentativas serem necessárias para ocorrer um sucesso é:
	\[\mathbb{P}(X=n) = (1-p)^{j-1}p\]

	a probabilidade de serem necessários \textit{n} insucessos antes do primeiro sucesso é:
	\[\mathbb{P}(X=n) = (1-p)^{j}p\]

\end{mdframed}

\subsubsection{Exercicios}

\begin{ex}
	Suponha que a probabilidade de um componente de computador ser defeituoso é de 0,2.  Numa mesa de testes, uma batelada é posta à prova, um a um.  Determine a probabilidade do primeiro defeito encontrado ocorrer no sétimo componente testado.
\end{ex}

\begin{sol}
	Estamos interessados em encontrar a probabilidade de serem necessários 6 insucessos até o primeiro sucesso, no caso na sétima tentativa.
	\[\mathbb{P}(7) = 0,2(1-0,2)^{7-1} = 0,0524\]
\end{sol}

\begin{ex}
	André deve a Renata R \$130,00. Cada viagem de Renata à casa de André custa R\$50,00 e a probabilidade de André ser encontrado em casa é $\frac{1}{3}$. Se Renata encontrar André, conseguirá cobrar a dívida.
	\begin{itemize}
		\item[a)] Qual a probabilidade de Renata ter de ir mais de 3 vezes à casa de André para conseguir cobrar a dívida? 
		\item[b)] Se na segunda vez em que Renata foi à casa de André ainda não o encontrou, qual a probabilidade de conseguir cobrar na 3a vez?
	\end{itemize}
\end{ex}

\begin{sol}
	\begin{itemize}
		\item[a)] Estamos interessados em descobrir $\mathbb{P}(X > 3) = 1 - (\mathbb{P}(0) + \mathbb{P}(1) + \mathbb{P}(2) + \mathbb{P}(3))$.
		
		
	\end{itemize}
\end{sol}

\subsection{ Distribuição Hipergeométrica}

Considere uma população com $ N $ objetos nos quais $ M $ são classificados como do tipo $ A $ e $ N-M $ são classificados como do tipo $ B $. Por exemplo, em um lote de $ 100 $ ($ N $) peças temos $ 10 $ ($ M $) peças defeituosas e $ 90 $ ($ N-M $) peças conformes. Tomamos uma amostra ao acaso, sem reposição e não ordenada de $ n $ objetos. Seja $ X $ a variável aleatória que conta o número de objetos classificados como do tipo $ A $ na amostra. Então a distribuição de probabilidade de $ X $ é dada por:

\[\mathbb{P}\left(X=k\right)=\frac{\left(\begin{array}{c}M\\k\end{array}\right)\left(\begin{array}{c}N-M\\n-k\end{array}\right)}{\left(\begin{array}{c}N\\n\end{array}\right)}\] 	

sendo $ k $ inteiro e $ \max\{0,n-(N-M)\}\leq k \leq \min\{M,n\} $.

\begin{df}
	Diremos que uma variável aleatória $ X $ tem distribuição hipergeométrica de parâmetros $ M $, $ N $ e $ n $ se sua função de probabilidade for dada da maneira acima. Denotamos $ X \sim \ \text{Hgeo}(M,N,n) $
\end{df}

\begin{eg}
	Seja $ X $ a variável aleatória que segue o modelo hipergeométrico com parâmetros $ N=10 $, $ M=5 $ e $ n=4 $. Determine a probabilidade $ \mathbb{P}(X\leq 1) $.
\end{eg}

\begin{sol}
	Para calcular a probabilidade procurada precisamos.

\[\mathbb{P}(X\leq 1)=\mathbb{P}(X=0)+\mathbb{P}(X=1)=\frac{\left(\begin{array}{c}5\\0\end{array}\right) + \left(\begin{array}{c}5\\4\end{array}\right)}{\left(\begin{array}{c}10\\4\end{array}\right)} + \frac{\left(\begin{array}{c}5\\1\end{array}\right)\left(\begin{array}{c}5\\3\end{array}\right)}{\left(\begin{array}{c}10\\4\end{array}\right)}=\frac{5}{210}+\frac{50}{210}\approx 0,2619.\]
\end{sol}

\begin{eg}
	Uma urna contém $ 10 $ bolas, das quais $ 6 $ são brancas e $ 4 $ pretas. Suponha que décimos retirar $ 5 $ bolas da urna qual a probabilidade de que das $ 5 $ bolas retiradas $ 3 $ sejam brancas?
\end{eg}

\begin{sol}
	Para este problema basta usarmos a distribuição hipergeométrica, com $ M=6 $, $ k=3 $, $ N=10 $ e $ n=5 $.

\[\mathbb{P}\left(X=3\right)=\frac{\left(\begin{array}{c}6\\3\end{array}\right)\left(\begin{array}{c}4\\2\end{array}\right)}{\left(\begin{array}{c}10\\5\end{array}\right)}=\frac{10}{21}.\]
\end{sol}


\section{Modelos Probabilisticos Contínuos}

\subsection{Distribuição Uniforme}

\begin{df}Uma variável aleatória $ X $ tem distribuição Uniforme no intervalo $ [a,b] $ se sua função densidade de probabilidade for dada por:
\end{df}

\[f(x)=\left\{\begin{array}{l} \frac{1}{b-a}, \ \hbox{se} \ a\leq x\leq b;\\ 0, \ \hbox{caso contrário}\end{array}\right.\] 	


\begin{obs}
	Dada uma função de probabilidade  $f(x)$, temos que 
	\[ \int_{0}^{\infty} f(x)dx = 1 \]
\end{obs}

\begin{eg}
	A ocorrência de panes em qualquer ponto de uma rede telefônica de $ 7 $ km foi modelada por uma distribuição Uniforme no intervalo $ [0, 7] $. Qual é a probabilidade de que uma pane venha a ocorrer nos primeiros $ 800 $ metros? E qual a probabilidade de que ocorra nos $ 3 $ km centrais da rede?
	\begin{sol}
		A função densidade da distribuição Uniforme é dada por $ f(x)=\frac{1}{7} $ se  $ 0\leq x\leq 7 $ e zero, caso contrário. Assim, a probabilidade de ocorrer pane nos primeiros 800 metros é

\[\mathbb{P}\left(X\leq 0,8\right)=\int_0^{0,8} f(x)dx=\frac{0,8-0}{7}=0,1142.\] 	

e a probabilidade de ocorrer pane nos 3 km centrais da rede é

\[\mathbb{P}\left(2\leq X\leq 5\right)=\int_2^5f(x)dx=\mathbb{P}\left(X\leq 5\right)-\mathbb{P}\left(X\leq 2\right)=5/7-2/7\approx 0,4285.\]
	\end{sol}
\end{eg}

\subsubsection{Função Geradora de Momentos, Valor Esperado e Variância.}

O valor esperado de uma variável aleatória $ X $ com distribuição uniforme é dado por

\[\mathbb{E}(X) = \frac{a+b}{2}.\]

\subsubsection{Exercícios}

\begin{ex}
	Várias linguagens de programação de computadores têm funções que geram números pseudo-aleatórios cuja distrubuição é basicamente uniforme. Se uma função desse tipo gera números entre 0 e 2, qual a probabilidade de um número gerado estar entre 1 e 1,5? Qual a média esperada e o respectivo desvio padrão ?
	
	\begin{sol}
		A função densidade da distribuição Uniforme é dada por $ f(x)=\frac{1}{2} $ se  $ 0\leq x\leq 2 $ e zero, caso contrário. Assim, a probabilidade de ocorrer pane nos primeiros 800 metros é:
		
		\[\mathbb{P}(1\leq X \leq 1,5) \]
	\end{sol}
\end{ex}

\section{Processos Estocásticos}

\end{document}