\documentclass[10pt,a4paper]{article}
\usepackage[utf8]{inputenc}
\usepackage{amsmath}
\usepackage{amsfonts}
\usepackage{amssymb}
\usepackage{mdframed}
\usepackage{amsthm}
\usepackage{setspace}
\usepackage[lastexercise]{exercise}

\author{Andrey França}

%theorems
\theoremstyle{definition}
\newtheorem*{eg}{Exemplo}
\newtheorem*{ex}{Exercício}
\newtheorem*{sol}{Solução}
\newtheorem*{df}{Definição}
\newtheorem*{obs}{Observação}
\newtheorem*{imp}{Implementação}

\newcounter{x}\setcounter{x}{1}
\newtheorem{inneraxiom}{Axiom}
\newenvironment{axiom}[1]
  {\renewcommand\theinneraxiom{\arabic{x} (#1)}\inneraxiom\stepcounter{x}}
  {\endinneraxiom}

%commands
%\renewcommand{\baselinestretch}{1.3} 


\title{Set Theory Notes}

\begin{document}
\maketitle
\tableofcontents
\newpage

\section{Noções de Lógica}
\begin{df}
	Chama-se proposição ou sentença toda oração declarativa que pode ser classificada de verdadeira ou falsa.
\end{df}

Observamos que toda proposição apresenta três características obrigatórias:

\begin{itemize}
	\item[1)] sendo oração, tem sujeito e predicado;
	\item[2)] é declarativa (não é exclamativa nem interrogativa)
	\item[3)] tem um e somente um, dos dois valores lógicos: ou é verdadeira(V) ou é falsa(T).
\end{itemize}

\begin{eg}
	São proposições:
	
	\item[a)] 9 $\neq$ 5
	\item[b)] 7 > 3
	\item[c)] 2 $\in \mathbb{Z}$
	\item[d)] 3 | 11
	\item[e)] $\mathbb{Z} \subset \mathbb{Q}$ 
\end{eg}

\section{Axioms  of Zermelo-Fraenkel}
\begin{axiom}{Axiom of Extensionality}
	If X and Y have the same elements, then X = Y.
\end{axiom}

\begin{axiom}{Axiom of Pairing}
	For any a and b there exists a set {a, b} that contains exactly a and b
\end{axiom}

\begin{axiom}{Axiom Schema of Separation}
	if P is a property (with parameter p), then for any X and p there exists a set Y = $\lbrace$u $\in$ : P(u,p)$\rbrace$ that contains all those u $\in$ X that have property P.
\end{axiom}

\begin{axiom}{Axiom of Union}
	For any X there exists a set Y = $\cup$X, the union of all elements of X.
\end{axiom}

\begin{axiom}{Axiom of Power Set}
	For any X there exists a set Y = P(X), the set of all subsets of X.
\end{axiom}

\begin{axiom}{Axiom of Infinity}
	There exists an infinite set.
\end{axiom}

\begin{axiom}{Axiom Schema of Replacement}
	If a class F is a function, the for any X there exists a set Y = F(X) = $\lbrace F(x) : x \in X \rbrace$
\end{axiom}

\end{document}