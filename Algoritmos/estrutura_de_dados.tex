\documentclass[10pt,a4paper]{article}

\usepackage[utf8]{inputenc}
\usepackage{amsmath}
\usepackage{amsfonts}
\usepackage{amssymb}
\usepackage{mdframed}
\usepackage{amsthm}
\usepackage{setspace}
\usepackage[lastexercise]{exercise}

\author{Andrey França}

%theorems
\theoremstyle{definition}
\newtheorem*{eg}{Exemplo}
\newtheorem*{ex}{Exercício}
\newtheorem*{sol}{Solução}
\newtheorem*{df}{Definição}
\newtheorem*{obs}{Observação}
\newtheorem*{imp}{Implementação}

%commands
%\renewcommand{\baselinestretch}{1.3} 


%% Informações do Documento
\author{Andrey França}
\title{Anotações Estrutura de Dados}
\date{\today}


\begin{document}

\maketitle

\section{Tabela Hash}

Um método de pesquisa com uso da transformação de chave é constituido de duas etapas principais:

\begin{itemize}

	\item[1.] Computar o valor da função de transformação (também conhecida como função 
	\textit{hashing}), a qual transforma a chave de pesqusia em um edereço na tabela;
	
	\item[2.] Considerando que duas ou mais chaves podem transformar em um mesmo endereço 
	da tabela, é necessário existir um método para lidar  com colisões.
	
\end{itemize}

\section{Processamento de Cadeia de Caracteres}


\end{document}