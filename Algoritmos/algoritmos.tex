\documentclass[10pt,a4paper]{article}

\usepackage[utf8]{inputenc}
\usepackage{amsmath}
\usepackage{amsfonts}
\usepackage{amssymb}
\usepackage{mdframed}
\usepackage{amsthm}
\usepackage{setspace}
\usepackage[lastexercise]{exercise}

\author{Andrey França}

%theorems
\theoremstyle{definition}
\newtheorem*{eg}{Exemplo}
\newtheorem*{ex}{Exercício}
\newtheorem*{sol}{Solução}
\newtheorem*{df}{Definição}
\newtheorem*{obs}{Observação}
\newtheorem*{imp}{Implementação}

%commands
%\renewcommand{\baselinestretch}{1.3} 


%% Informações do Documento
\author{Andrey França}
\title{Anotações Algoritmos}
\date{\today}
\begin{document}

\maketitle

\begin{quote}

	"Ao verificar que um dado programa está muito lento, uma pessoa prática pede uma 
	máquina mais rápida ao seu chefe, mas o ganho potencial que uma máquina mais 
	rápida pode proporcionar é tipicamente limitado por um fator de 10 por razões 
	técnicas ou econômicas. Para obter um ganho maior, é preciso buscar melhores 
	algoritmos. Um bom algoritmo, mesmo rodando em uma máquina lenta,sempre acaba 
	derrotando (para instâncias grandes do problema) um algoritmo pior rodando em 
	uma máquina rápida. Sempre."
	
	\begin{flushright}
		-S. S. Skiena, The Algorithm Desig
	\end{flushright}
	
\end{quote}

\section{Introdução}

Na área de análise de algoritmos, existem dois tipos de problemas bem distintos, conforme apontou Knuth (1971):

\begin{itemize}

	\item[(i)] \textbf{Análise de um algoritmo particular:} Qual é um custo de um dado algoritmo
	 para resolver um problema específico? Neste  caso características importantes do
	  algoritmo em questão devem ser investigadas.  Geralmente faz se a análise do 
	  número de vezes em que cada parte do algoritmo é executada.
	  
	 \item[(ii)] \textbf{Análise de uma classe de algoritmos:} Qual é o algoritmo de menor custo
	  possível para resolver um problema em particular? Analisa-se uma família de algoritmos
	   e escolhe-se o mais adequado para realizar aquele problema em particular.
	   
\end{itemize}


O custo de utilização de um algoritmo pode ser medido de várias maneiras. Uma delas é mediante
 execução do programa em um computador real, sendo o tempo de execução medido diretamente.

\textbf{Desvantagens}:

\begin{itemize}
	
	\item[(i)] os resultados são dependentes do compilador;
	
	\item[(ii)] os resultados dependem de hardware;
	
	\item[(iii)] quando grandes quantidades de memória são utilizadas, as medidas de tempo 
	podem depender desse aspecto;
\end{itemize}

Apesar disso, em algumas situações são favoráveis utilizar medidas reais, por exemplo, quando 
existem vários algoritmos distintos para resolver o mesmo tipo de problema, todos com um custo
de execução dentro da mesma ordem de grandeza.

Uma forma mais adequada de medir o custo de utilização de um algoritmo é por meio do uso de 
um modelo matemático baseado em um computador idealizado, por exemplo, o computador MIX 
proposto por Knuth(1968). 

\end{document}