\documentclass[10pt,a4paper]{article}

\usepackage[utf8]{inputenc}
\usepackage{amsmath}
\usepackage{amsfonts}
\usepackage{amssymb}
\usepackage{mdframed}
\usepackage{amsthm}
\usepackage{setspace}
\usepackage[lastexercise]{exercise}

\author{Andrey França}

%theorems
\theoremstyle{definition}
\newtheorem*{eg}{Exemplo}
\newtheorem*{ex}{Exercício}
\newtheorem*{sol}{Solução}
\newtheorem*{df}{Definição}
\newtheorem*{obs}{Observação}
\newtheorem*{imp}{Implementação}

%commands
%\renewcommand{\baselinestretch}{1.3} 


\title{Anotações Algoritmos}

\begin{document}
\maketitle
\tableofcontents
\newpage

\section{Introdução}
Informalmente, um algoritmo é qualquer procedimento computacional bem definido que toma algum valor ou conjunto de valores como entrada e produz algum valor ou conjunto de valores como saída. Portanto, um algoritmo é uma sequência de passos computacionais que transformam a entrada na saída.\\\\
\textbf{Estrutura de Dados}

Uma estrutura de dados é um meio para armazenar e organizar dados com o objetivo de facilitar o acesso ou as modificações. Nenhuma estrutura de dados única funciona bem para todos os propósitos, e assim é importante conhecer os pontos fortes e as limitações de várias delas.

\subsection{Eficiência}
\section{Estrutura de Dados}

\subsection{Listas}

\begin{df}
	Uma lista é um conjunto de nós, cada um desses nós armazenam um objeto ou chave (inteiros, reais, tipos definidos pelo programador etc.) e uma referência para o próximo nó.
\end{df}


\subsubsection{Lista com Vetor}
\subsubsection{Lista Encadeada}
\subsubsection{Lista Duplamente Encadeada}

\subsection{Pilha}
\subsubsection{Pilha com Vetor}
\subsubsection{Pilha Encadeada}

\subsection{Fila}
\subsubsection{Fila com Vetor}
\subsubsection{Fila Encadeada}

\subsection{Arvores}
\subsubsection{Arvore Binária}
\subsubsection{Arvore Binária de Busca}
\subsubsection{Arvore AVL}
\subsubsection{Arvore B}


\section{Algoritmos de Ordenação}
Algoritmos de ordenação são algoritmos que colocam elementos de uma lista em uma certa ordem. Algoritmos de ordenação eficientes são importantes para otimizar o uso de outros algoritmos (como busca por exemplo). Algoritmos de ordenção são geralmente  classificados por:

\begin{itemize}
	\item[•] Complexidade;
	\item[•] Memória;
	\item[•] Estabilidade;
\end{itemize}


\subsection{Selection Sort}
A ordenação por seleção (do inglês, selection sort) é um algoritmo de ordenação baseado em se passar sempre o menor valor do vetor para a primeira posição (ou o maior dependendo da ordem requerida), depois o de segundo menor valor para a segunda posição, e assim é feito sucessivamente com os \textit{n-1} elementos restantes, até os últimos dois elementos.

\begin{ex}
	Escreva uma função que verifique se um vetor v[0..n-1] está em ordem crescente.
	\begin{eg}
		Para isto basta criar uma função que percorre todo vetor \code{arr} verificando se o elemento \code{arr[i + 1] < arr[i]}, caso o encontre algum elemento que não se encaixe nessa comparação, então retornaremos falso, isto é, o vetor não esta em orden crescente.
	\end{eg}
\end{ex}
\begin{lstlisting}[language=C++, caption={Verificar se um vetor está em ordem crescente}]
bool verifica_ordem(int arr[], int len){
	for (int i = 0; i < len; i++){
		if (arr[i] < arr[i+1]) continue;
		else return false;
	}   			
	return true;
}
		\end{lstlisting}
\subsubsection{Complexidade}
O Selection Sort compara a cada interação um elemento com os outros, visando encontrar o menor. Dessa forma, podemos entender que não existe um melhor caso mesmo que o vetor esteja ordenado ou em ordem inversa serão executados os dois laços do algoritmo, o externo e o interno. A complexidade deste algoritmo será sempre $ \mathcal{O}(n^{2})$ enquanto que, por exemplo, os algoritmos Heapsort e Mergesort possuem complexidades $\mathcal{O}(n\log{}n)$.

\subsubsection{Implementação}

\begin{lstlisting}[language=C++, caption={Selection Sort implemetação}]
void selection_sort(int num[], int tam) { 
	int i, j, min, aux;
  	for (i = 0; i < (tam-1); i++) 
  	{
     	min = i;
     	for (j = (i+1); j < tam; j++) {
       		if(num[j] < num[min]) 
         	min = j;
     	}
     	if (i != min) {
       		aux = num[i];
       		num[i] = num[min];
       		num[min] = aux;
     	}
  	}
}
\end{lstlisting}
\subsection{Insertion Sort}
\subsection{Bubble Sort}
\subsection{Merge Sort}
\subsection{Quick Sort}
\end{document}