\documentclass[10pt,a4paper]{article}

\usepackage[utf8]{inputenc}
\usepackage{amsmath}
\usepackage{amsfonts}
\usepackage{amssymb}
\usepackage{mdframed}
\usepackage{amsthm}
\usepackage{setspace}
\usepackage[lastexercise]{exercise}

\author{Andrey França}

%theorems
\theoremstyle{definition}
\newtheorem*{eg}{Exemplo}
\newtheorem*{ex}{Exercício}
\newtheorem*{sol}{Solução}
\newtheorem*{df}{Definição}
\newtheorem*{obs}{Observação}
\newtheorem*{imp}{Implementação}

\newcounter{x}\setcounter{x}{1}
\newtheorem{inneraxiom}{Axiom}
\newenvironment{axiom}[1]
  {\renewcommand\theinneraxiom{\arabic{x} (#1)}\inneraxiom\stepcounter{x}}
  {\endinneraxiom}

%commands
%\renewcommand{\baselinestretch}{1.3} 


\title{Anotações Algoritmos}

\begin{document}
\maketitle
\tableofcontents
\newpage

\section{Estrutura de Dados}

\subsection{Listas}

\begin{df}
	Uma lista é um conjunto de nós, cada um desses nós armazenam um objeto ou chave (inteiros, reais, tipos definidos pelo programador etc.) e uma referência para o próximo nó.
\end{df}


\subsubsection{Lista com Vetor}
\subsubsection{Lista Encadeada}
\subsubsection{Lista Duplamente Encadeada}

\subsection{Pilha}
\subsubsection{Pilha com Vetor}
\subsubsection{Pilha Encadeada}

\subsection{Fila}
\subsubsection{Fila com Vetor}
\subsubsection{Fila Encadeada}

\subsection{Arvores}
\subsubsection{Arvore Binária}
\subsubsection{Arvore Binária de Busca}
\subsubsection{Arvore AVL}
\subsubsection{Arvore B}


\section{Algoritmos de Ordenação}
\subsection{Selection Sort}
\subsection{Insertion Sort}
\subsection{Bubble Sort}
\subsection{Merge Sort}
\subsection{Quick Sort}
\end{document}